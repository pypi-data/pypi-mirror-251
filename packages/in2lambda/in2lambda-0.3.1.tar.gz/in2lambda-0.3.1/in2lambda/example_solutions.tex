\documentclass[12pt]{article}

% Mechanics course definitions and macros.
%\usepackage{mechanics}
\usepackage{graphicx}

% Add space between paragraphs instead of indenting.
\usepackage{parskip}

\addtolength{\topmargin}{-4em}
\addtolength{\evensidemargin}{-3em}
\addtolength{\oddsidemargin}{-3em}
\addtolength{\textheight}{6em}
\addtolength{\textwidth}{4em}

%\usepackage{bm}
\usepackage{graphics}
\usepackage{color}
\usepackage{amsmath}

\begin{document}
\section*{Mechanics Answer Sheet 4} 

\begin{enumerate}
  \item
  \begin{enumerate}

  \item Momentum is conserved. Kinetic energy is not conserved. (The
    collision is said to be \textbf{inelastic}.)

  \item Immediately after the collision, the block and projectile have
    merged into a single object of mass $M+m$ with total momentum $mu$.
    The speed $U$ of the merged block and projectile is given by
    % 
    \begin{displaymath}
      (M + m)U = mu \qquad \Rightarrow \qquad U = \frac{mu}{M+m} ,
    \end{displaymath}
    %
    and its kinetic energy $K$ is
    %
    \begin{displaymath}
      K = \frac{1}{2}(M+m)U^2 = \frac{m^2 u^2}{2(M+m)}.
    \end{displaymath}
    %
    At the maximum height $h$, all of this kinetic energy has become
    gravitational potential energy:
    %
    \begin{displaymath}
      (M+m)gh =  \frac{m^2 u^2}{2(M+m)}.
    \end{displaymath}
    %
    Hence
    %
    \begin{displaymath}
      h = \frac{m^2 u^2}{2(M+m)^2 g}.
    \end{displaymath}

  \item The question says that $m = 12\,$g = $0.012\,$kg, $M = 10\,$kg,
    and $h = 10\,$cm $= 0.1\,$m. Rearranging the result of part (b)
    gives
    %
    \begin{displaymath}
      u = \frac{M+m}{m} \sqrt{2gh}
          = \frac{10.012}{0.012} \sqrt{2 \times 9.8 \times 0.1} \approx
          1168 \, \text{m}\cdot\text{s}^{-1} \approx 1.17\,\text{km}\cdot\text{s}^{-1}.
    \end{displaymath}

  \end{enumerate}
  
\item Suppose the woman of mass $60\,$kg starts at the origin
  ($X_{\text{woman}} = 0$) standing on the left-hand end of the plank.
  The centre of mass of the $10\,$m long plank of mass $20\,$kg is at
  $X_{\text{plank}} = 5\,$m, and the centre of mass of the joint system
  (woman $+$ plank) is at
  \[
    X = \frac{60 \times X_{\text{woman}} + 20 \times X_{\text{plank}}}
    {60 + 20} = 1.25\,\text{m}.
  \]

  There are no externally applied horizontal forces acting on the woman
  $+$ plank (treated as one system), so the position $X$ of the centre
  of mass cannot change. The diagram below, which pictures the woman and
  plank before and after she walks from one end to the other, shows that
  she moves a total distance of $2X = 2.5\,$m relative to the frozen
  surface of the lake.

\item
  \begin{enumerate}

  \item Using the formula
    $\displaystyle \frac{1}{\mu} = \frac{1}{m_1} + \frac{1}{m_2}$ gives
    $\mu = 2.4\,$kg.

  \item Differentiating the definition of the centre of mass,
    %
    \begin{displaymath}
      \vec{\boldsymbol{R}} = \frac{m_1 \vec{\boldsymbol{r}}_1 + m_2 \vec{\boldsymbol{r}}_2}{m_1 + m_2} ,
    \end{displaymath}
    %
    gives a formula for the velocity of the centre of mass:
    %
    \begin{displaymath}
      \dot{\vec{\boldsymbol{R}}} = \frac{m_1 \vec{\boldsymbol{v}}_1 + m_2 \vec{\boldsymbol{v}}_2}{m_1 + m_2}.
    \end{displaymath}
    %
    We are told that $\vec{\boldsymbol{r}}_1 = (1 + 3t)\boldsymbol{\hat{i}} + t\boldsymbol{\hat{j}}$ and
    $\vec{\boldsymbol{r}}_2 = -2t\boldsymbol{\hat{i}} + 5\boldsymbol{\hat{j}}$, so $\vec{\boldsymbol{v}}_1 = \dot{\vec{\boldsymbol{r}}}_1 = 3\boldsymbol{\hat{i}} + \boldsymbol{\hat{j}}$ and
    $\vec{\boldsymbol{v}}_2 = \dot{\vec{\boldsymbol{r}}}_2 = -2\boldsymbol{\hat{i}}$. Hence
    \begin{displaymath}
      \dot{\vec{\boldsymbol{R}}} = \frac{4(3\boldsymbol{\hat{i}} + \boldsymbol{\hat{j}}) + 6(-2\boldsymbol{\hat{i}})}{4 + 6} = 0.4\boldsymbol{\hat{j}}
      \;\text{m}\cdot\text{s}^{-1}.
    \end{displaymath}

  \item The velocity of $m_1$ relative to $m_2$ is
    \begin{displaymath}
      \vec{\boldsymbol{v}}_1 - \vec{\boldsymbol{v}}_2 = (3\boldsymbol{\hat{i}} + \boldsymbol{\hat{j}}) - (-2\boldsymbol{\hat{i}}) = (5\boldsymbol{\hat{i}} + \boldsymbol{\hat{j}})\,\text{m}\cdot\text{s}^{-1}.
    \end{displaymath}

  \item The momentum of $m_1$ in the centre-of-mass frame is
    \begin{displaymath}
      \vec{\boldsymbol{p}}_1^{\ast} = m_1(\vec{\boldsymbol{v}}_1 - \dot{\vec{\boldsymbol{R}}}) = 4(3\boldsymbol{\hat{i}} + 0.6\boldsymbol{\hat{j}})
      = (12\boldsymbol{\hat{i}} + 2.4\boldsymbol{\hat{j}}) \, \text{kg}\cdot\text{m}\cdot\text{s}^{-1}.
      \end{displaymath}

  \item The velocities of the two particles in the centre-of-mass frame
    are:
    \begin{align*}
      \vec{\boldsymbol{v}}_1^{\ast} &= \vec{\boldsymbol{v}}_1 - \dot{\vec{\boldsymbol{R}}} = (3\boldsymbol{\hat{i}} +
                     0.6\boldsymbol{\hat{j}})\,\text{m}\cdot\text{s}^{-1}, \\
      \vec{\boldsymbol{v}}_2^{\ast} &= \vec{\boldsymbol{v}}_2 - \dot{\vec{\boldsymbol{R}}} = (-2\boldsymbol{\hat{i}} -
                     0.4\boldsymbol{\hat{j}})\,\text{m}\cdot\text{s}^{-1}.
    \end{align*}
    %
    Their kinetic energies are
    %
    \begin{align*}
      \frac{1}{2}m_1 (v_1^{\ast})^2 &= \frac{1}{2} 4 \left ( 3^2 +
                                      (0.6)^2 \right ) =
                                      18.72\,\text{J}, \\
      \frac{1}{2}m_2 (v_2^{\ast})^2 &= \frac{1}{2} 6 \left ( 2^2 +
                                      (0.4)^2 \right ) =
                                      12.48\,\text{J}.
    \end{align*}
    %
    The total kinetic energy in the centre-of-mass frame is $31.2\,$J.

  \end{enumerate}

\item As in the answer to Q2(b), the velocity of the centre of mass can
  be found by differentiating the equation for $\vec{\boldsymbol{R}}(t)$:
  %
  \begin{displaymath}
    \dot{\vec{\boldsymbol{R}}} = \frac{d}{dt} \left (
      \frac{\sum_{i=1}^{N} m_i \vec{\boldsymbol{r}}_i}{\sum_{i=1}^{N} m_i}
    \right ) = \frac{\sum_{i=1}^{N} m_i \vec{\boldsymbol{v}}_i}{\sum_{i=1}^{N} m_i}
    = \frac{\sum_{i=1}^{N} m_i \vec{\boldsymbol{v}}_i}{M}
  \end{displaymath}
  %
  where $M \equiv \sum_{i=1}^{N} m_i$. Since the velocity
  $\vec{\boldsymbol{v}}_i^{\ast}$ of mass $m_i$ in the centre-of-mass frame is given by
  $\vec{\boldsymbol{v}}_i - \dot{\vec{\boldsymbol{R}}}$, its momentum
  $\vec{\boldsymbol{p}}_i^{\ast} = m_i(\vec{\boldsymbol{v}}_i - \dot{\vec{\boldsymbol{R}}})$. The total momentum in the
  centre-of-mass frame is
  %
  \begin{align*}
    \sum_{i=1}^{N} \vec{\boldsymbol{p}}_i^{\ast}
    &=  \sum_{i=1}^{N} m_i (\vec{\boldsymbol{v}}_i - \dot{\vec{\boldsymbol{R}}} ) \\
    &= \sum_{i=1}^{N} m_i \vec{\boldsymbol{v}}_i - \dot{\vec{\boldsymbol{R}}} \sum_{i=1}^{N} m_i \\
    &= M \left ( \frac{\sum_{i=1}^{N} m_i \vec{\boldsymbol{v}}_i}{M} - \dot{\vec{\boldsymbol{R}}}
      \right ) \\
    &= 0. \qquad\qquad
      \text{(Note that this is a zero \underline{vector}.)}
  \end{align*}
  %
  The final step used the equation for $\dot{\vec{\boldsymbol{r}}}$ above.

\item The total laboratory-frame kinetic energy of the $N$ particles
  (masses $m_i$, velocities $\vec{\boldsymbol{v}}_i$) is
  %
  \begin{displaymath}
    K = \sum_{i=1}^{N} \frac{1}{2} m_i v_i^2 .
  \end{displaymath}
  %
  The laboratory (unstarred) and centre-of-mass (starred) velocities of
  the particles are related by $\vec{\boldsymbol{v}}_i^{\phantom{}} = \vec{\boldsymbol{v}}_i^{\ast} +
  \dot{\vec{\boldsymbol{R}}}$, where
  %
  \begin{displaymath}
    \dot{\vec{\boldsymbol{R}}} = \frac{\sum_{i=1}^{N} m_i \vec{\boldsymbol{v}}_i}{\sum_{i=1}^{N} m_i}
  \end{displaymath}
  %
  is the velocity of the centre of mass. Hence
  %
  \begin{align*}
    K &= \frac{1}{2} \sum_{i=1}^{N} m_i^{\phantom{}} 
        \left ( \vec{\boldsymbol{v}}_i^{\ast} + \dot{\vec{\boldsymbol{R}}} \right )^2 \\
      &= \frac{1}{2} \sum_{i=1}^{N} m_i^{\phantom{}} \left [  
        \left ( v_i^{\ast} \right)^2 + 
        2 (\vec{\boldsymbol{v}}_i^{\ast}\cdot \dot{\vec{\boldsymbol{R}}}) + \dot{R}^2 \right ] \\
      &= \frac{1}{2} \sum_{i=1}^{N} m_i^{\phantom{}} 
        \left ( v_i^{\ast}\right )^2  
        + \left ( \sum_{i=1}^{N} m_i^{\phantom{}} \vec{\boldsymbol{v}}_i^{\ast} \right ) 
        \cdot \dot{\vec{\boldsymbol{R}}}
        + \frac{1}{2} \left ( \sum_{i=1}^{N} m_i^{\phantom{}} \right ) 
        \dot{R}^2 \\
      &= \frac{1}{2}\sum_{i=1}^{N} m_i^{\phantom{}} 
        \left ( v_i^{\ast} \right )^2 + \frac{1}{2} M \dot{R}^2 
        \qquad (\text{where } M \equiv \sum_{i=1}^{N} m_i^{\phantom{}}) \\
      &= K^{\ast} + \frac{1}{2} M \dot{R}^2 \\
      &= \text{(KE in centre-of-mass frame)} + \text{(KE of centre of mass)}.
  \end{align*}
  %
  The cross term vanished because $\sum_{i=1}^{N} m_i^{\phantom{}}
  \vec{\boldsymbol{v}}_i^{\ast}$ is the total momentum in the centre-of-mass frame, which
  Q2 showed is always zero.

\item Start by writing down the definition of the coefficient of
  restitution and the law of conservation of momentum in the case when
  $u_2 = 0$:
  %
  \begin{align*}
    v_2 - v_1 &= eu_1 ,\\
    m_1 v_1 + m_2 v_2 &= m_1 u_1 .
  \end{align*}
  %
  Our job is to solve for $v_1$ and $v_2$ as functions of $u_1$. One way
  is to multiply the first equation by $m_1$,
  %
  \begin{displaymath}
    m_1 v_2 - m_1 v_1 = e m_1 u_1,
  \end{displaymath}
  %
  and add it to the second equation to obtain
  %
  \begin{displaymath}
    m_2 v_2 + m_1 v_2 = m_1 u_1 + e m_1 u_1 ,
  \end{displaymath}
  % 
  which rearranges to give
  %
  \begin{displaymath}
    v_2 = \frac{(1 + e)m_1 u_1}{m_1 + m_2}.
  \end{displaymath}
  %
  Plugging this back into $v_1 - v_2 = -eu_1$ yields
  %
  \begin{displaymath}
    v_1 = v_2 - eu_1 = \frac{(1+e)m_1 u_1}{m_1 + m_2} - \frac{e(m_1 +
    m_2)u_1}{m_1 + m_2} = \frac{(m_1 - e m_2)u_1}{m_1 + m_2}.
  \end{displaymath}


\item Options (a) and (g) are true.

  The ping-pong ball of mass $m$ is so much lighter than the basketball
  of mass $M$ that its momentum almost reverses when it collides:
  $m\vec{\boldsymbol{u}}_m \rightarrow -m\vec{\boldsymbol{u}}_m$. Since the total momentum $m\vec{\boldsymbol{u}}_m$ is
  conserved, the momentum of the basketball after the collision must be
  approximately $2m\vec{\boldsymbol{u}}_m$. The absolute value of the momentum of the
  basketball after the collision is larger than that of the ping-pong
  ball.

  The momentum of the basketball after the collision is approximately
  $2m\vec{\boldsymbol{u}}_m$, so its velocity $\vec{\boldsymbol{v}}_M \approx \frac{2m}{M} \vec{\boldsymbol{u}}_m$. The
  final kinetic energies of the basketball and ping-pong ball are:
  %
  \begin{align*}
    K_M
    &= \frac{1}{2}M |\vec{\boldsymbol{v}}_M|^2 \approx \frac{1}{2} M
      \left |\frac{2m\vec{\boldsymbol{u}}_m}{M}\right |^2 = \frac{2m^2}{M}
      |\vec{\boldsymbol{u}}_m|^2 ,\\
    K_m
    &\approx \frac{1}{2} m |-\vec{\boldsymbol{u}}_m|^2 .
  \end{align*}
  %
  The ratio $K_m/K_M = M/(4m)$ is much bigger than 1 (the mass of one
  basketball is much larger than the mass of four ping-pong balls), so
  the ping-pong ball has much more kinetic energy than the basketball.

  \item
    \begin{enumerate}

      \item The total momentum before the collision that forms the
        complex is
        \begin{displaymath}
          p_{\text{tot}} = m_{\text{NO}} u_{\text{NO}} + m_{\text{O}_3}
          u_{\text{O}_3} = (14 + 16)m_u u_{\text{NO}} + (3\times 16)m_u u_{\text{O}_3},
        \end{displaymath}
        where $m_u \approx 1.66 \times 10^{-27}\,$kg is the atomic mass
        unit. Setting the value of $u_{\text{NO}}$ to $-550\,$m$\cdot$
        s$^{-1}$ and $u_{\text{O}_3} = +550\,$m$\cdot$s$^{-1}$ gives
        %
        \begin{displaymath}
          p_{\text{tot}} \approx 1.64 \times 10^{-23}\,\text{kg}\cdot\text{m}\cdot\text{s}^{-1}.
        \end{displaymath}

        The total momentum is conserved during the collision in which
        the two molecules bind to form the [NO--O$_3$]$^{\ast}$ complex,
        so
        %
        \begin{displaymath}
          p_{\text{[NO--O$_3$]}^{\ast}} \approx 1.64 \times
          10^{-23}\,\text{kg}\cdot\text{m}\cdot\text{s}^{-1}
        \end{displaymath}
        %
        and the velocity of the complex is
        %
        \begin{displaymath}
          v_{\text{[NO--O$_3$]}^{\ast}} \approx \frac{1.64 \times
            10^{-23}}{[(14 + 16) + (3 \times 16)] m_u} \approx
          127\,\text{m}\cdot\text{s}^{-1}.
        \end{displaymath}

      \item The total kinetic energy before the collision is
        %
        \[
          K = \frac{1}{2} m^{}_{\text{NO}}v_{\text{NO}}^2 + \frac{1}{2}
              m^{}_{\text{O}_3} v_{\text{O}_3}^2 
            = \frac{1}{2} [ (14 + 16) + (3 \times 16) ] m_u (550)^2 
            \approx 1.96\times 10^{-20}\,\text{J}.
          \]

        The total kinetic energy after the collision is
        %
        \[
          K' = \frac{1}{2} m^{}_{\text{[NO--O$_3$]}^{\ast}} \,
               v_{\text{[NO--O$_{3}$]}^{\ast}}^2 \\
             = \frac{1}{2} (78 m_u) (127)^2 \\
             \approx 1.04 \times 10^{-21}\,\text{J}.
        \]

        The total KE has reduced so the collision was inelastic.

        The KE lost has gone into the internal energy of the activated
        complex, which has higher chemical energy than the two molecules
        of which it is formed (it is only metastable) plus internal
        vibrational and rotational energy.

      \item The collision will only form the products if the internal energy
        of the complex exceeds the activation barrier. The internal
        energy is
        %
        \begin{displaymath}
          E^{\text{internal}}_{[\text{NO--O}_3]^{\ast}} = K - K' \approx
          1.96 \times 10^{-20} - 1.04 \times 10^{-21}
          \approx 1.85 \times 10^{-20}\,\text{J}.
        \end{displaymath}
        %
        An activation barrier of of $9.6\,\text{kJ}\cdot\text{mol}^{-1}$
        translates into
        %
        \begin{displaymath}
          \frac{9.6 \times 10^3}{6.022 \times 10^{23}} \approx 1.59 \times
          10^{-20} \, \text{J per complex}.
        \end{displaymath}
        %
        This is less than the internal energy provided by the collision,
        so the NO$_2$ and O$_2$ products will be formed (some of the
        time, at least).

      \item Set the potential energy of interaction to zero when the two
        oxygen atoms are far apart. Since the atoms are approaching each
        other, they already have some kinetic energy even before the
        attractive part of the interaction starts accelerating them
        towards each other. This means that their total energy $E$ is
        greater than zero. If we choose to work in the centre-of-mass
        frame, the total momentum is zero and the two oxygen atoms
        approach each other with equal and opposite momentum.

        Momentum and total energy are conserved, so the O$_2$ molecule
        must have zero total momentum but positive total energy $E$. If
        the molecule is to form, the kinetic energy of the molecules
        arriving at their equilibrium spacing (the difference between
        $E$ and the minimum value of $U(r)$) must be transferred into
        internal vibrational, rotational and electronic degrees freedom
        of the O$_2$ molecule.

        Since the total energy of the molecule is greater than zero, it
        is large enough to provide the two O atoms with enough kinetic
        energy to unbind and escape to infinity. Unless some other
        nearby atom or molecule takes away the excess energy before it
        is too late, the highly-excited oxygen molecule will soon break
        up.

        \begin{center}
          \includegraphics[width=0.55\textwidth]{figures/O2_potential.png}
        \end{center}

        The figure above illustrates the situation. It shows the
        potential energy $U(r)$ of the two O atoms as a function of
        their separation $r$ (compare with Q5 of the Mechanics part of
        the week 6 problem sheet). The total energy $E$, indicated by
        the dashed red line, is greater than zero. The total kinetic
        energy at separation $r$, indicated by the green arrow, is $E -
        U(r)$. Because $E > 0$, the two escaping atoms will keep moving
        away from each other even as $r \rightarrow \infty$.
        
  \end{enumerate}
  
    \begin{enumerate}
    
    \item Using conservation of energy, the speed $v_0$ of the ball when
      it first hits the ground is related to the initial height $h_0$ by
      $mgh_0 = \frac{1}{2}mv_0^2$. This gives
      %
      \begin{displaymath}
        v_0 = \sqrt{2gh_0}.
      \end{displaymath}
      %
      (Could also have used the SUVAT equation $v^2 = u^2 + 2as$.) Since
      $v_0 = g t_0$,
      %
      \begin{displaymath}
        t_0 = \sqrt{\frac{2h_0}{g}}.
      \end{displaymath}

    \item The upward speed after the first bounce is $v_1 = ev_0$. The
      height reached after the first bounce is
      %
      \begin{displaymath}
        h_1 = \frac{v_1^2}{2g} = \frac{e^2 v_0^2}{2g}
        = e^2 h_0.
      \end{displaymath}
      %
      The time taken to reach the top of the first bounce is
      %
      \begin{displaymath}
        t_1 = \sqrt{\frac{2h_1}{g}} = et_0.
      \end{displaymath}
      %
      The height $h_n$ and drop time $t_n$ for the $n^{\text{th}}$ bounce
      are
      %
      \begin{displaymath}
        h_n = e^{2n} h_0, \qquad t_n = e^n t_0.
      \end{displaymath}
    
    \item The initial drop takes time $t_0$. The duration of the
      $n^{\text{th}}$ bounce is $2t_n$, not $t_n$, because the ball has to
      rise and then drop again. The total time $T$ for the ball to stop
      bouncing is the sum of the durations of the initial drop and all
      subsequent bounces:
      %
      \begin{align*}
        T &= t_0 + 2 e t_0 + 2 e^2 t_0 + 2 e^3 t_0 + \ldots \\
          &= 2 t_0 (1 + e + e^2 + e^3 + \ldots) - t_0 \\
          &= \frac{2 t_0}{1 - e} - t_0
        \qquad \text{(using the formula given in the question)}\\
          &= \frac{1 + e}{1 - e} \, t_0.
      \end{align*}

      If $h_0 = 5\,$m, $e = 0.5$, and $g = 10\,\text{m}\cdot\text{s}^{-2}$,
      then
      %
      \begin{displaymath}
        t_0 = \sqrt{\frac{2 \times 5}{10}} = 1\,\text{s},
      \end{displaymath}
      %
      and the time $T$ taken to stop bouncing is
      %
      \begin{displaymath}
        T = \frac{1 + 0.5}{1 - 0.5} \, t_0 = 3\,\text{s}.
      \end{displaymath}

    \end{enumerate}

\item The diagram below shows that the change in momentum of the
  rocket+fuel in time $dt$ can be found as follows:
  %
  \begin{align*}
    \text{change in momentum}
    &= \text{(momentum at time $t + dt$)} - \text{(momentum at time $t$)} \\
    &=\big( (M + dM)(v + dv) + (-dM)(v-u) \big) - \big( Mv \big)\\
    &= M dv + u dM . \qquad \text{(ignoring second-order terms)}
  \end{align*}

  \begin{center}
    \includegraphics[width=0.45\textwidth]{figures/rocket-momentum.png}
  \end{center}
  
  Since the only external force acting on the system is gravity, this
  must be equal to the gravitational impulse $-Mg\,dt$ (which is negative
  because gravity acts against the rocket's acceleration):
  %
  \begin{align*}
    &M dv + u dM = -Mg \, dt \\
    \Rightarrow \qquad
    &dv = -u\frac{dM}{M} - g \, dt \\
    \Rightarrow \qquad
    &\int_{0}^{v(t)} dv' = -u \int_{M_0}^{M(t)} \frac{dM'}{M'} -
      g\int_0^{t} dt' \\
    \Rightarrow \qquad
    &\left [ v' \right ]_0^{v(t)} = -u \left [ \ln M' \right
      ]_{M_0}^{M(t)} - g \left [ t' \right ]_0^{t} \\
    \Rightarrow \qquad
    &v(t) = -u \left \{ \ln(M(t)) - \ln(M_0) \right \} - gt \\
    \Rightarrow \qquad
    &v(t) = u\ln\left ( \frac{M_0}{M(t)} \right ) - gt,
  \end{align*}
  as required.

\item The free-space rocket equation is
  %
  \begin{displaymath}
    v(t) = v_0 + u \ln \left ( \frac{M_0}{M(t)} \right ),
  \end{displaymath}
  %
  where $v_0$ is the initial speed of the rocket, $u$ is the exhaust
  speed, $M_0$ is the rocket's initial mass (including any fuel it is
  carrying), and $M(t)$ is its mass at time $t$ (including any fuel not
  yet burnt).

  For the first stage of the journey, call the initial mass
  $M_1^{\text{init}}$. The initial speed is zero. The initial mass of
  fuel is $3M_1^{\text{init}}/4$, but only $2/3$ of this is burnt during
  stage 1. The mass $M_1^{\text{final}}$ of the rocket and remaining
  fuel at the end of stage 1 is therefore
  $M_1^{\text{init}} - (2/3)(3M_1^{\text{init}}/4) =
  M_1^{\text{init}}/2$. We can now work out the speed $v_1$ at the end
  of stage 1:
  %
  \begin{displaymath}
    v_1 = u \ln \left ( \frac{M_1^{\text{init}}}{M_1^{\text{final}}} \right )
    = u\ln \left ( \frac{M_1^{\text{init}}}{M_1^{\text{init}}/2} \right ) = u\ln 2.
  \end{displaymath}

  Once the first stage of the rocket has detached, the remaining mass of
  fuel, which we know is equal to
  $(1/3)(3M_1^{\text{init}}/4) = M_1^{\text{init}}/4$, accounts for
  $4/5$ of the mass of the rocket. This tells us the initial mass
  of stage 2:
  %
  \begin{displaymath}
    M_2^{\text{init}} = (5/4)(M_1^{\text{init}}/4) =
    5M_1^{\text{init}}/16.
  \end{displaymath}
  %
  At the end of stage 2, the fuel has all been burnt, so
  $M_2^{\text{final}} = M_2^{\text{init}} - M_1^{\text{init}}/4 =
  M_1^{\text{init}}/16$. The velocity $v_2$ at the end of stage 2 is
  %
  \begin{align*}
    v_2 &= v_1 + u \ln \left (
      \frac{M_2^{\text{init}}}{M_2^{\text{final}}} \right ) 
    = u\ln 2 + u\ln \left
      ( \frac{5M_1^{\text{init}}/16}{M_1^{\text{init}}/16} \right ) \\
    &= u\ln 2 + u\ln 5 = u\ln 10.
  \end{align*}

  Suppose we had instead made the journey with a single-stage rocket of
  initial mass $M_1^{\text{init}}$ and the same initial mass of fuel,
  $3M_{1}^{\text{init}}/4$. At the end of the single stage, all of the
  fuel has been burnt, so the final mass is
  $M_1^{\text{init}} - 3M_{1}^{\text{init}}/4 = M_1^{\text{init}}/4$.
  The final speed $v_1'$ is given by
  %
  \begin{align*}
    v_1' = u\ln\left ( \frac{M_1^{\text{init}}}{M_1^{\text{init}}/4}
    \right ) = u\ln 4,
  \end{align*}
  %
  which is approximately $0.6u\ln(10)$.

    \end{enumerate}
    

\end{document}